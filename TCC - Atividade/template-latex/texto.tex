\documentclass[12pt]{article}

\usepackage{sbc-template}
\usepackage{graphicx,url}
\usepackage[utf8]{inputenc}
\usepackage[brazil]{babel}
\usepackage[latin1]{inputenc}  

     
\sloppy

\title{Revisão Bibliográfica}

\begin{document} 

\maketitle

\begin{abstract}
This work aims to carry out a bibliographic review on some articles related to the main theme of the project to be developed. The articles are organized in a table where they are described for which platform they were designed, which tool was used to implement the chatbot and what the main objective of the project is.
\end{abstract}
     
\begin{resumo} 
Este trabalho tem como objetivo realizar uma revisão bibliográfica sobre alguns artigos relacionados ao tema principal do projeto a ser desenvolvido. Os artigos são organizados em uma tabela onde são descritas para qual plataforma foram projetados, que ferramenta foi utilizada para implementar o chatbot e qual o objetivo principal do projeto.
\end{resumo}


\section{Introdução}

A cada novo semestre, alunos novatos ingressam em uma instituição de ensino superior e encontram um ambiente educacional com estruturas diferentes do que estão acostumados. É muito comum que estudantes possam se sentir desinformados sobre o funcionamento do novo ambiente e busquem orientações de terceiros. 

Entretanto, os alunos acabam recorrendo a grupos de aplicativos de {\itshape chat} para obter informações não oficiais, muitas vezes desobedecendo regras gerais do grupo ou até mesmo não conseguindo ingressar por conta de um limite máximo de usuários ou não obtenção do {\itshape link} de acesso. Além disso, as perguntas dos estudantes costumam ser repetitivas e possuem um certo padrão, o que permite buscar uma solução para automatizar esse processo.

{\itshape Chatbots} (do inglês {\itshape chat robots}, ou “robôs de conversação”) são agentes de conversação ou aplicativos inteligentes que simulam a conversação como se fossem seres humanos \cite{lucchesi:18}. Atualmente, {\itshape chatbots} são muito utilizados em empresas para auxiliar na comunicação, em trocas e devoluções de produtos e atendimento aos clientes, mas seu uso não se limita somente para o ramo empresarial. Qualquer pessoa pode criar um {\itshape chatbot} para conversar, gerenciar informações e executar operações, realizar pesquisas de opinião, jogar ou até mesmo controlar dispositivos de Internet das Coisas (IoT, na sigla em inglês {\itshape Internet of Things}).

Tendo em vista suas diversas aplicações, foi planejado a utilização de um {\itshape chatbot} no meio acadêmico com o intuito de responder algumas das dúvidas mais frequentes do corpo discente da instituição. É esperado que o uso dessa aplicação diminua a dependência do estudante para recorrer a grupos de {\itshape chat} e servidores para obter informações repetitivas, além de facilitar a comunicação do estudante por canais oficiais de comunicação.

O objetivo geral do projeto é a elaboração de um {\itshape chatbot} voltado para responder perguntas frequentes e tirar dúvidas de estudantes da instituição de ensino. Os objetivos específicos são: 


\begin{itemize}
  \item Realizar um levantamento acerca das perguntas mais frequentes em grupos de {\itshape chat} acadêmico;
  \item Implementação de um chatbot utilizando a linguagem Python;
  \item Criação de uma identidade visual para o {\itshape bot}, assim como definição de características de personalidade para o mesmo;
  \item Embarcar o programa em um Raspberry Pi.
\end{itemize}

A hipótese levantada é que {\itshape chatbots} poderiam orientar os alunos acerca de assuntos voltados ao espaço acadêmico. Os estudantes poderiam recorrer aos mesmos para responder perguntas ao qualquer momento, sem a exigência de seguir regras de moderação ou cadastro em {\itshape softwares} ou {\itshape sites} específicos.

A justificativa para essa hipótese é o estudo de sistemas similares que obtiveram uma boa resposta de estudantes e servidores no uso de {\itshape chatbots} voltados para o meio acadêmico, seja no campo educacional ou em orientações acerca do funcionamento da instituição. Os projetos desenvolvidos oferecem espaço para evolução das aplicações, assim como a exploração do Raspberry Pi como ferramenta para criação de agentes conversacionais.

\section{ Método de Pesquisa }

Para construir a base de informações que será utilizada pelo {\itshape bot}, será realizada uma pesquisa quantitativa em duas etapas acerca das principais dúvidas entre os estudantes da instituição. A primeira etapa consiste na coleta e classificação das perguntas mais frequentes em um grupo de {\itshape chat} do aplicativo de mensagens {\itshape Whatsapp} durante um determinado período de tempo. Essa classificação levará em consideração o período do semestre vigente na instituição.

A segunda etapa é composta pela aplicação de um questionário para os estudantes. O objetivo desse questionário é o levantamento de informações acerca do curso em que o aluno está matriculado, o semestre que o mesmo está cursando e quais dúvidas são as mais frequentes dentro das categorias definidas na etapa anterior. Além disso, o questionário permitirá que o estudante adicione novas dúvidas relacionadas ao espeço acadêmico. Dessa forma, novas categorias de perguntas poderiam ser identificadas e acrescentadas à base de dados do {\itshape chatbot}.

\section{Cronograma} \label{sec:firstpage}

O cronograma para a implementação do projeto será dividido em semanas, concentrando-se na elaboração da pesquisa, evolução do protótipo e testes com a lógica de resposta do {\itshape chatbot}. A tabela 1 mostra o cronograma elaborado para a etapa de implementação e evolução do sistema, relacionando as principais atividades com o tempo estimado para realizar cada tarefa.


\begin{table}[h!]
\caption{Cronograma}
\label{table:1}
\begin{tabular}{ |p{8cm}||p{6cm}|  }
 \hline
Atividade & Período\\
 \hline
 Coleta e classificação de perguntas frequentes & 25/06/21 a 27/06/21  \\
 Elaboração do questionário & 02/07/21 a 04/07/21  \\
 Aplicação do questionário & 05/07/21 a 12/07/21  \\
 Análise dos resultados & 12/07/21 a 16/07/21  \\
 Elaboração do banco de perguntas & 16/07/21 a 18/07/21  \\
 Definição das respostas do {\itshape bot} & 16/07/21 a 18/07/21  \\
 Evolução do código & 19/07/21 a 23/07/21 \\
 Testes do sistema & 23/07/21 a 25/07/21  \\
 \hline
\end{tabular}
\end{table}

\section{ Referencial Teórico}

O trabalho elaborado em \cite{maciel:19} realiza uma pesquisa entre diversas plataformas disponíveis para escolher aquela que melhor se adequa às especificidades do projeto e a partir disso definem uma plataforma para implementar Maciel, um {\itshape chatbot} para ser utilizado na Universidade Federal do Ceará (UFC) campus Russas. O chatbot foi criado com a ferramenta {\itshape Dialogflow} e integrado ao aplicativo de mensagens {\itshape Whatsapp}, recebendo uma avaliação positiva entre os usuários e melhorias após os testes iniciais.

O trabalho realizado por \cite{bulhoes:20} propõe a criação de um {\itshape chatbot} para auxiliar estudantes no processo de interpretação textual. O agente conversacional é implementado com uma interface gráfica em Java, realizando perguntas a partir de um texto que é exibido ao leitor. O programa faz uso de um interpretador para processar o texto digitado pelo usuário e consultar a melhor resposta na base de dados. Dentre os 50 alunos que utilizaram o programa, 48 afirmaram que as perguntas feitas pelo {\itshape chatbot} contribuíram de forma relevante para o entendimento do texto.


O projeto em \cite{lucchesi:18} apresenta a criação do agente conversacional “Metis”, projetado para conversar com alunos como apoio às atividades de educação à distância. Foi realizado um estudo qualitativo acerca do funcionamento dos {\itshape chatbots} e seu potencial uso na educação. A implementação do agente utilizou o sistema A.L.I.C.E. no qual organiza a base de conhecimento em XML. A análise dos resultados mostrou que houve um aumento, ainda que moderado, da média geral alcançada pelas turmas que utilizaram o agente para auxiliar seus estudos.

O trabalho de \cite{araujo:20} propõe o desenvolvimento de um {\itshape chatbot} para responder perguntas/dúvidas de alunos da disciplina de estrutura de dados, seguindo a ementa definida para a disciplina ministrada na Universidade Federal do Ceará - Campus Russas. Foi desenvolvido um {\itshape website}, um serviço em Java para permitir a comunicação entre o site e o serviço Watson da IBM, e a manutenção das lógicas de diálogo e informações pelo {\itshape Watson Assistant}. Apesar da maioria dos usuários declarar que o sistema apresenta dificuldades de uso, cerca de 76,8\% dos alunos  vêem benefícios na utilização de um {\itshape chatbot} para auxiliar suas atividades   .

O trabalho proposto em \cite{catbot:19} tem como intuito desenvolver um {\itshape chatbot} para o meio acadêmico, visando facilitar e tornar mais rápido o acesso às informações mais relevantes relacionadas ao Instituto de Computação e seus cursos, em especial ao curso de Sistemas de Informação. Foi criada uma personalidade e interface gráfica para o {\itshape chatbot}. Além disso, foi utilizado o {\itshape Watson Assistant} para processar a lógica de respostas da aplicação, assim como executar os testes com os usuários. O programa teve uma boa recepção dos usuários, que ofereceram diversas sugestões para melhorar o projeto.

A Tabela 1 relaciona os trabalhos analisados com suas principais carcterísticas de desenvolvimento de {\itshape chatbots}. A utilização dos {\itshape chatbots} possuem objetivos "educacionais" quando são projetados para auxiliar os estudos dos estudantes em uma determinada disciplina, e "orientações" quando o objetivo é fornecer informações acerca do espaço acadêmico.

\begin{table}[h!]
\caption{Trabalhos similares e principais características}
\label{table:1}
\begin{tabular}{ |p{3cm}||p{3cm}|p{3cm}|p{3cm}|  }
 \hline
Autor & Plataforma & Ferramenta & Utilização\\
 \hline
 \cite{araujo:20}   & Web    & {\itshape Watson Assistant} &   educacional\\
 \cite{bulhoes:20}&   Desktop  & {\itshape Program AB}   & educacional\\
 \cite{catbot:19} & Web & {\itshape Watson Assistant} &  orientações\\
 \cite{lucchesi:18}    & Web & A.L.I.C.E. & educacional\\
 \cite{maciel:19}  & Mobile  & {\itshape Dialogflow} & orientações\\
 \hline
\end{tabular}
\end{table}




%\section{Materiais e Métodos} \label{sec:firstpage}

%[...]

%\section{Resultados e discussões}

%[...]

%\section{Considerações Finais} 
%[...]

\bibliographystyle{sbc}
\bibliography{bibliot}

\end{document}
